\chapter{Methodology}
\label{chap:meth}
  
\section{Software and Hardware Details} \label{sec:m1}
  
\subsection{Non-functional requirements} \label{sec:m1.1}

Non-functional requirements are requirements for the environment that the application
will be run in, such as operating system, or details concerning the hardware.
Android
1. WiFi connection to the internet
2. A mobile device running with Android (at least version 5.0)

Computer
1. Windows Vista, 7, 8 or 10 operating system
2. Wireless internet connection
3. VR Set Available and connected
4. Unity Setup with Steam VR

\subsection{Functional requirements} \label{sec:m1.2}

Functional requirements are often based on user stories detailed lists of features that
are going to be implemented in the application. They tend to be really helpful for further
application design.

\section{Use Stories} \label{sec:m2}
User stories are usually sets of very overall ideas for what users might want to have in
the applications. They are usually the base for creating more advanced specification for
the application.

\subsection{Use Cases} \label{sec:m2.1}
Text 1

Text 2

Text 3

\subsection{How make a Life Span Recorder?} \label{sec:m2.2}
Text 1

Text 2

Text 3

\section{Use Cases} \label{sec:m3}

\subsection{Use Cases} \label{sec:m3.1}
Text 1

Text 2

Text 3

\subsection{How make a Life Span Recorder?} \label{sec:m3.2}
Text 1

Text 2

Text 3

\section{Principles of Mobile App Development} \label{sec:m4}


\section{Identifying required Data} \label{sec:m5}
