\chapter*{Abstract}
\addcontentsline{toc}{chapter}{Abstract}
\label{chap:abstract}

    Your memory is your mind's documenting framework. It contains everything that you have learnt. You can store an astounding measure of data. From multiple points of view, our memories shape our identity. They make up our internal biographies. hey tell us who we're connected to, who we've touched amid our lives, and who has touched us. To put it plainly, our recollections are vital to the embodiment of our identity as people. 


	Moreover, human computer interaction techniques are rapidly evolving lately affecting our daily lives enormously and effectively. Some techniques where proven to have a greater impact on our daily routines than others. This paper aims at finding a new technique using both virtual reality and deployed mobile application that could help improve humans’ memory, testing the effect of the technique and discussing the impact it reflected on their normal daily routines.


	This technique is based on the famous method of loci. The method for loci is a strategy for memory enhancement which utilises visualisations with the use of spatial memory, recognisable data about one's condition, to rapidly and productively review data. The method of loci is also known as the memory journey, memory palace, or mind palace technique.


	The project is composed of 2 main blocks. The first block is an iOs mobile application which collects places, pictures, people and logged lifetime events and classifies them according to the method of loci. The second block is a memory palace built to display all the collected info and is viewed in a VR set where the user is able to walk around and visualise all his memories to improve his long term memory.

